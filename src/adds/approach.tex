
\subsection{Алгоритм}

\begin{frame}{Приложение: алгоритм поиска нечетких дубликатов}
    \begin{itemize} \footnotesize
        \item $\nv$\ ---~новое видео;
        \item $\setsv = \{\sv[1], \sv[2], \dots, \sv[n]\}$\ ---~исходные видео:
            \begin{itemize}
                \item[${\color{zdarkblue}\leftarrow}$]
                    {\scriptsize  $\setsv$ может быть пустым};
                \item[${\color{zdarkblue}\leftarrow}$]
                    {\scriptsize для непустого $\setsv$
                        вычислены дескрипторы сцен элементов.}
            \end{itemize}
        \item[1.] Сравниваем дескриптор каждой
            сцены $\videoshot_{\color{red}\nv,i}$ из $\nv$\
            с дескриптором каждой сцены
            $\videoshot_{\color{red}\sv[k],j}$ из $\sv[k]$ в $L_2$.
        \item[2.] Если дескрипторы совпали $\nv$ c дескрипторами $\sv[k]$.
            на~некотором временном промежутке ,
            то считаем эту часть $\nv$\ ---~дубликатом $\sv[k]$,
            \begin{itemize}
                \item[] {\scriptsize несовпавшие
                    части $\nv$\ помещаем в $\setsv$}.
            \end{itemize}
        \item[3.] Если дескрипторы не совпали, то считаем $\nv$\ уникальным и
            добавляем в $\setsv$.
    \end{itemize}
\end{frame}
