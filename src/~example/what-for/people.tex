\subsection{Люди}

\begin{frame}{Зачем идти на прикладную математику}
    \begin{center}
            \tikzstyle{peoplef} = [
                draw=green!50!black!70,thick,
                minimum height=3cm,
                minimum width=3cm,
                top color=yellow!20,
                bottom color=green!60!black!20,
                decorate,decoration={
                    random steps,segment length=2pt,amplitude=3pt
                }
            ]
            \tikzstyle{studyf} = [
                draw=red!50!black!70,thick,
                minimum height=3cm,
                minimum width=3cm,
                top color=yellow!20,
                bottom color=red!60!black!20,
                decorate,decoration={
                    random steps,segment length=2pt,amplitude=3pt
                }
            ]
            \tikzstyle{jobf} = [
                draw=blue!50!black!70,thick,
                minimum height=3cm,
                minimum width=3cm,
                top color=yellow!20,
                bottom color=blue!60!black!20,
                decorate,decoration={
                    random steps,segment length=2pt,amplitude=3pt
                }
            ]
            \begin{tikzpicture}[
                thick,node distance=3.8cm,
                text height=2.7ex,text depth=.5ex,
                auto
            ]
                \node[peoplef] (people) {
                    \SansRoundedLightC \color{green!50!black!90}
                    \begin{tabular}{c}
                        {\Huge $ \color{green!50!black!90} \lambda $} \\
                        \\
                        интересные \\
                        люди \\
                    \end{tabular}
                };
                \node[studyf, right of=people] (study) {
                    \SansRoundedLightC  \color{red!50!black!90}
                    \begin{tabular}{c}
                         {\Huge $ \color{red!50!black!90} \omega $} \\
                         \\
                        интересное \\
                        обучение \\
                    \end{tabular}
                };
                \node[jobf, right of=study] (job) {
                    \SansRoundedLightC \color{blue!50!black!90}
                    \begin{tabular}{c}
                        {\Huge $ \color{blue!50!black!90} \rho $} \\
                        \\
                        интересная \\
                        работа \\
                    \end{tabular}
                };
            \end{tikzpicture}
            { \Huge
                \[
                    {\color{green!50!black!90} \lambda } \Rightarrow
                    {\color{red!50!black!90} \omega } \Rightarrow
                    {\color{blue!50!black!90} \rho }
                \]
            }
    \end{center}
\end{frame}

