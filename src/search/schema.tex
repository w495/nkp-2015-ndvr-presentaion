
\subsection{Схема}

\begin{frame}{Общая схема поиска по видео}
    \sf\scriptsize
    \tikzstyle{videof} = [
        rectangle, rounded corners,
        thick,minimum size=3em,
        draw=black!85,
        top color=black!5,
        bottom color=black!5
    ]
    \tikzstyle{methf} = [
        rectangle, rounded corners,
        thick,
        minimum width =1em,
        minimum height =2em,
        draw=black!85,
        top color=black!5,
        bottom color=black!5,
        text width=8em,
        align=center,
    ]
    \tikzstyle{layerf} = [
        rectangle,dashed,
        thick,
        minimum width =1em,
        minimum height = 1em,
        inner sep=0.5em,
        draw=black!75,
        decorate,decoration={random steps,segment length=0.5em,amplitude=0.1em}
    ]
    \tikzstyle{dbf} = [
        cylinder,  shape border rotate=90,
        aspect=0.25,
        thick,
        minimum size=2em,
        draw=black!85,
        top color=black!5,
        bottom color=black!5
    ]
    \tikzstyle{qresf} = [
        rectangle,
        thick,
        minimum width =2em,
        minimum height =1.5em,
        draw=black!85,
        top color=black!5,
        bottom color=black!5,
        text width=5em,
        align=center,
        decorate,decoration={random steps,segment length=0.2em,amplitude=0.05em}
    ]
    \tikzstyle{userf} = [
        rounded  rectangle,
        thick,
        minimum width =2em,
        minimum height=2em,
        draw=black!85,
        top color=black!5,
        bottom color=black!5,
        align=center,
    ]
    \tikzstyle{arrowf} = [
        -latex,thick,black!85,
    ]
    \tikzstyle{arrowr} = [
        -latex,thick,dotted,pacificgray,
    ]
    \begin{center}
        \begin{tikzpicture}[very thick,node distance=5em]
            \node[videof] (video) {\small Видео};
            \begin{scope}
                \node[methf, below of=video]    (cshots)  {Съёмки};
                \begin{scope}[node distance=10em]
                    \node[methf, left of=cshots]    (cframes) {Кадры};
                    \node[methf, right of=cshots]   (cscenes) {Сцены};
                \end{scope}
                \begin{scope}[on background layer]
                    \node[layerf, fit=(cframes) (cshots) (cscenes), draw] (composition) {};
                    \draw [arrowf] (video) -- (composition)
                        node [midway, right] {\color{teal}Сегментация};
                \end{scope}
            \end{scope}
            \begin{scope}
                \node[methf, below of=composition]  (fmotion)   {Движение};
                \begin{scope}[node distance=10em]
                    \node[methf, left of=fmotion]       (fframes)   {Ключевые кадры};
                    \node[methf, right of=fmotion]      (fobjects)  {Объекты};
                \end{scope}
                \begin{scope}[on background layer]
                    \node[layerf, fit=(fobjects) (fmotion) (fframes), draw] (features) {};
                    \draw [arrowf] (composition) -- (features)
                                node [midway, right] {\color{teal}Извлечение признаков};
                \end{scope}
            \end{scope}
            \begin{scope}
                \node[methf, below of=fmotion]      (annotation)     {Аннотиро\-вание};
                \begin{scope}[node distance=10em]
                    \node[methf, left of=annotation]    (datamining)     {Анализ};
                    \node[methf, right of=annotation]   (classification) {Классифи\-кация};
                \end{scope}
                \begin{scope}[on background layer]
                    \node[layerf, fit=(annotation) (datamining) (classification), draw] (processing) {};
                    \draw [arrowf] (features) -- (processing)
                        node [midway, right] {\color{red}Построение сигнатур};
                \end{scope}
            \end{scope}
            \begin{scope}[node distance=5em]
                \node[dbf, below of=annotation]      (index)     {Индекс};
                \draw [arrowf, rounded corners=2em] (processing.west) --  ++(-2em,0) |-  (index.west)
                    node [near end, above, align=center, text width=6em]
                        {\color{teal}Смысловая индексация};
                \draw [arrowf, rounded corners=2em] (features.east) --  ++(2em,0) |-  (index.east)
                    node [near end, above, align=center, text width=6em]
                        {\color{teal}Индексация данных};
            \end{scope}
            \begin{scope}[node distance=5em]
                \node[qresf, below left  of=index] (query)   {Запрос};
                \node[qresf, below right of=index] (results) {Выдача};
                \draw [arrowf] (index) to [bend left=20] (results);
                \draw [arrowf] (query) to [bend left=20] (index);
                \node[userf, below right of=query] (user) {\small  Зритель};
                \draw [arrowf] (user)    to [bend left=20] (query);
                \draw [arrowf] (results) to [bend left=20] (user);

                \draw [arrowr, rounded corners=2em] (query.west) --  ++(-12em,0) |-  (video.west);

            \end{scope}
        \end{tikzpicture}
    \end{center}
\end{frame}
