

\subsection{Сегментация}

% Съёмка — набор кадров внутри непрерывной временной области, кадры, которой
% значительно отличаются от кадров соседних областей. Кадры съёмок связаны между собой
% по смыслу. Съёмки считаются основной структурной единицей видео. Исторически,
% съёмка (монтажный план, кинематографический кадр) — отрезок киноплёнки, на котором
% запечатлено непрерывное действие между пуском и остановкой камеры, или между двумя
% монтажными склейками. Методы сегментации видео работают с уже созданным видео, и
% достоверно определить была монтажная склейка или нет, не всегда возможно. Именно по
% этой причине в литературе используется конструктивное определение съёмки, которое мы
% привели в начале абзаца.
%

%
% Сцена — группа смежных съёмок, связанных конкретной темой или предметом.
% С точки зрения семантики, самым мелким элементом видео является кадр
% (фотографический кадр, кадрик).
% Съёмка является более крупным делением.
% Из съёмок складываются сцены, а из сцен видео целиком.
%


\begin{frame}{Сегментация видео}

    \begin{center}
        \includegraphics[width=7cm]{img/video/scene-shot-frame.pdf}
    \end{center}


\end{frame}
